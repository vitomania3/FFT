\documentclass[11pt, oneside, final]{article} \sloppy 
\usepackage[utf8]{inputenc} 
\usepackage{a4wide} 
\usepackage[russian]{babel} 
\usepackage{graphicx} 
\usepackage{epstopdf} 
\usepackage{amsmath} 
\usepackage{amsfonts} 
\usepackage{amssymb} 
\usepackage{amsthm}

\numberwithin{equation}{section} 
\newtheorem{definition}{Определение}[section] 
\newtheorem{theorem}{Теорема}[section] 
\newtheorem{property}{Свойство}[section] 
\newtheorem{corollary}{Следствие}[theorem] 
\newtheorem{lemma}[theorem]{Лемма} 
\newtheorem*{statement}{Утверждение} 
\renewenvironment{proof}{
\noindent\textit{Доказательство: }} {\qed}

%commands
\newcommand \bitem[1][]{
\item \textbf{#1}} 
\newcommand \four[1][\lambda]{\mathfrak{F}(#1)} 
\newcommand \fft[1][\lambda]{F(#1)} 
\newcommand {\intinf}[2] {\int \limits_{- \infty}^{+ \infty}{#1 \, d#2}}
\renewcommand \qed{$\blacksquare$}

\DeclareMathOperator{\sgn}{sgn}

\begin{document}

%Title
\thispagestyle{empty}
\begin{center}
    \ \vspace{-3cm}
    
    \includegraphics[width=0.5
    \textwidth]{msu}\\
    {\scshape Московский государственный университет имени М.~В.~Ломоносова}\\
    Факультет вычислительной математики и кибернетики\\
    Кафедра системного анализа
    
    \vfill
    
    {\LARGE Отчёт по практикуму}
    
    \vspace{1cm}
    
    {\Huge\bfseries "<Быстрое преобразование Фурье">} 
\end{center}

\vspace{1cm}
\begin{flushright}
    \large \textit{Студент 315 группы}\\
    В.\,В.~Кожемяк
    
    \vspace{5mm}
    
    \textit{Руководители практикума}\\
    к.ф.-м.н., доцент И.\,В.~Рублёв \\
    к.ф.-м.н., доцент П.\,А.~Точилин 
\end{flushright}

\vfill
\begin{center}
    Москва, 2017 
\end{center}
\pagebreak

%Contents
\tableofcontents

\pagebreak

%Task


\section{Постановка~задачи}

\subsection{Общая~формулировка~задачи} Дана система функций (всюду далее, если не сказано противное, предполагается, что $ f(t) : \mathbb{R} \rightarrow \mathbb{R} $ и функция суммируема и обладает достаточной гладкостью) 
\begin{equation}\label{functions} 
    \left\{ 
    \begin{aligned}
        f_1(t) &= \dfrac{1 - \cos^2{t}}{t} \\
        f_2(t) &= te^{-2t^2} \\
        f_3(t) &= \dfrac{2}{1 + 3t^6} \\
        f_4(t) &= e^{-5|t|}\ln({3 + t^4})
    \end{aligned}
    \right. 
\end{equation}
Для каждой функции из системы \eqref{functions} требуется:
\begin{enumerate}
    \item Получить аппроксимацию преобразования Фурье $ F(\lambda) $ при помощи быстрого преобразования Фурье ({\bfseries БПФ / FFT}), выбирая различные шаги дискретизации исходной функции и различные окна, ограничивающие область определения $ f(t) $.
    \item Построить графики $ F(\lambda). $
    \item Для функций $ f_1(t) $ и $ f_2(t) $ из заданного набора вычислить аналитически преобразование Фурье 
    \begin{equation}\label{fourier_transform} 
        F(\lambda) = \intinf {f(t) e^{-i\lambda t}} {t}
    \end{equation}
    и сравнить графики $ F(\lambda) $, полученные из аналитического представления и из аппроксимации через {\bfseries БПФ}.\\
\end{enumerate}

%Formal task

\subsection{Формальная~постановка~задачи} 
\begin{enumerate}
    \item Реализовать на языке MATLAB функцию \\\texttt{plotFT(hFigure,~fHandle,~fFTHandle,~step,~inpLimVec,~outLimVec)} \\со следующими параметрами: 
    \begin{itemize}
        \bitem[hFigure] ~--- указатель на фигуру, в которой требуется отобразить графики \bitem[fHandle] ~--- указатель на функцию (\texttt{Function Handle}), которую требуется преобразовывать ($ f(t) $) \bitem[fFTHandle] ~--- указатель на функцию (\texttt{Function Handle}), моделирующую аналитическое преобразование Фурье \eqref{fourier_transform} функции $ f(t) $ (может быть пустым вектором, в таком случае график аналитического преобразования строить не требуется) \bitem[step] ~--- положительное число, задающее шаг дискретизации $\Delta t $ \bitem[inpLimVector] ~--- вектор-строка, задающая окно $ [a, b] $ для функции $ f(t) $, первый элемент вектора содержит $ a $, второй $ b $, причём~$ a < b $, но не обязательно $ a = -b $ \bitem[outLimVector] ~--- вектор-строка, задающая окно $ [c, d] $ \textit{для вывода} графика преобразования Фурье (пределы~осей~абсцисс). В случае, если передаётся пустой вектор, следует брать установленные в фигуре пределы или определять свои разумным образом 
    \end{itemize}
    Данная функция строит графики вещественной и мнимой частей численной аппроксимации преобразования Фурье (\ref{fourier_transform}) функции $ f(t) $, заданной в \textbf{\texttt{fHandle}} (и, при необходимости, соответствующие графики~аналитического~преобразования~Фурье~$ F(\lambda) $) \\
    Кроме того, данная функция, должна возвращать структуру, содержащую следующие параметры: 
    \begin{itemize}
        \bitem[nPoints] ~--- число вычисляемых узлов сеточной функции, рассчитываемое по формуле:
        $ nPoints = \Bigl\lfloor {\dfrac{(b - a)}{step}}\Bigr\rfloor $
        \bitem[step] ~--- поправленное значение шага дискретизации $ \Delta t $, рассчитываемое по формуле:
        $ step= \dfrac{(b - a)}{nPoints} $
        \bitem[inpLimVec] ~--- окно $ [a, b] $ для функции \(f(t)\) \bitem[outLimVec] ~--- окно для вывода графика преобразования Фурье $ f(\lambda) $ 
    \end{itemize}
    \item Построить, используя написанную функцию \texttt{plotFT}, для каждой из функций системы \eqref{functions} графики $ \lambda $ для разных значений входных параметров (окон \textbf{inpLimVec, outLimVec} и частоты дискретизации \textbf{step}). \\В частности, для некоторых функций подобрать параметры так, чтобы проиллюстрировать эффекты \textit{наложения спектра, появления~ряби и их устранения} (в случае ряби~--- в точках непрерывности $ \lambda $) 
    \item Для функций $ f_1(t) $ и $ f_2(t) $ из системы \eqref{functions} вычислить аналитически их преобразования Фурье $ F(\lambda) $ и построить их графики вместе с графиками численной аппроксимации $ F(\lambda) $ 
\end{enumerate}

%fourier


\section{Вычисление~аналитических~преобразований~Фурье}

\subsection{Некоторые~необходимые~обозначения~и~соотношения} Напомним, что преобразование Фурье $ F(\lambda) $ функции $ f(t) $ задаётся формулой \eqref{fourier_transform}: 
\begin{align*}
    F(\lambda) = \intinf {f(t) e^{-i\lambda t}} {dt} 
\end{align*}
Далее, для краткости будем писать:
$$ f(t) \rightarrow F(\lambda) $$

\noindent Для вычисления аналитического преобразования Фурье нам потребуется лемма Жордана и теорема о вычетах. 

\begin{lemma}[лемма Жордана]
	Если $ f(z) $ аналитична в верхней полуплоскости $ Im(z) > 0 $, за исключением конечного числа изолированных особых точек, и при $ |z| \rightarrow \infty $ стремится к нулю равномерно относительно $ \arg z $. Тогда для любого $ \alpha > 0 $  $$ \lim\limits_{R \rightarrow \infty}\int\limits_{C_R}{f(z) \cdot e^{i \alpha z} \, {dz} } = 0 $$, \\
	где $ C_R $ верхняя полуокружность $ |z| = R, Im(z) > 0 $
	
\end{lemma}

\begin{theorem}[Основная теорема о вычетах]
	Пусть функция $ f(z) $ аналитична всюду в области $ D $ за исколючением конечного числа изолированных особых точек $ z_1, \cdots, z_n $. Тогда для любой замкнутой области $ \overline{G} $, лежащей в $ D $ и содержащей точки $ z_1, \cdots, z_n $ внутри, справедливо равенство
	
$$ \int\limits_{\partial G}{f(\zeta) \, d \zeta} = 2 \pi i \sum \limits_{k = 1}^{n} {res f(z_k)}$$
\end{theorem}

\noindent И еще необходим будет интеграл:
\begin{equation} \label{Gaussian_Integral}
	\int \limits_{0}^{+ \infty} {e^{-x^2} \, dx = \dfrac{\sqrt{\pi}} {2}}
\end{equation}

%Fourier 1

\subsection{Вычисление~аналитического~преобразования~Фурье\\функции~\(f_1(t) = \frac{1 - \cos^2{t}}{t} \)}
Для начала упростим выражение $$ f_1(t) = \dfrac{1 - \cos^2{t}}{t} = \dfrac{\sin^2{t}}{t} = \dfrac{1 - \cos{2t}}{2t}  $$
Преобразование Фурье $ F_1 (\lambda) $ функции $ f_1(t) $ задаётся формулой:
$$ F_1 (\lambda) = \intinf {\dfrac{1 - \cos{2t}}{2t} \cdot e^{-i\lambda t}} {dt} $$

\noindent Вычислим этот этот интеграл. Для начала распишем $ e^{-i \lambda t} $ по определению:

$$ e^{-i \lambda t} = {\cos{\lambda t}} - i{\sin{\lambda t}}. $$

\noindent Далее это выражение подставляем в интеграл и пользуемся линейностью интеграла:

$$  \intinf {\dfrac{1 - \cos{2t}}{2t} \cdot e^{-i\lambda t}} {dt} =  {\intinf {\dfrac{1 - \cos{2t}}{2t} \cdot \cos{\lambda t}} {dt}} - i{ \intinf {\dfrac{1 - \cos{2t}}{2t} \cdot \sin{\lambda t}} {dt}} = $$ 

$$= {\intinf {\dfrac{1}{2t} \cdot \cos{\lambda t}} {dt}} - {\intinf {\dfrac{\cos{2t}}{2t} \cdot \cos{\lambda t}} {dt}} - {i{\intinf {\dfrac{1}{2t} \cdot \sin{\lambda t}} {dt}]}} + i{\intinf {\dfrac{\cos{2t}}{2t} \cdot \sin{\lambda t}} {dt}} = $$ 

$$= \dfrac{1}{2}I_1 - \dfrac{1}{2}I_2 - \dfrac{i}{2}iI_3 + \dfrac{i}{2}iI_4 $$ \\

Рассмотрим каждый из интегралов отдельно.\\

\begin{enumerate}
	\item Вычисление $ I_1 $:
	
	Из \textit{леммы Жордана} (см. выше) и \textit{основной теоремы вычетов} следует, что
	\begin{equation}		
		{\int \limits_{- \infty}^{+ \infty} {e^{i \alpha 	x} \cdot f(x) \, dx}} = {2 \pi i \sum \limits_{k = 1}^{n} {res  [e^{i \alpha z} f(z) | z = Im(z_k) > 0]}}
	\end{equation}
	Поэтому интеграл $ I_1 $ равен нулю.
	
	\item Вычисление $ I_2 $:
	
	Заметим, что $$ \cos{2t} \cdot \cos{\lambda t} = \dfrac{\cos{(2 + \lambda)t} + \cos{(2 - \lambda)t}}{2}.$$
	Пользуясь линейностью интегралов и делая замену переменной, мы сведем эти интегралы к $ I_1. $
	
	\item Вычисление $ I_3 $:
	
	Следует рассматривать 3 случая:

	\begin{enumerate}
		\item $ \lambda > 0 $:	\\	
	Разбивая на два интеграла и делая замену переменной, мы сведем $ I_3 $ к уже известному интегралу (\ref{Gaussian_Integral}).
		\item $ \lambda < 0 $:	\\	
	Т.к. функция $ \sin{x} $ --- нечетная, значит знак "минус" можно вынести. И далее опять разбивая на два интеграла и делая замену переменной, мы сведем $ I_3 $ к уже известному интегралу (\ref{Gaussian_Integral}).
		\item $ \lambda = 0 $:	\\	
	Очевидно, что $ I_3 = 0$.
	\end{enumerate}
	
	\item Вычисление $ I_4 $:
	
	Заметим, что $$ \cos{2t} \cdot \sin{\lambda t} = \dfrac{\sin{(\lambda + 2)t} + \sin{(\lambda - 2)t}}{2}.$$
	 Тогда получаем
	
	$$ \int \limits_{- \infty}^{+ \infty} {\dfrac{\cos{2t} \sin{\lambda t}}{t} \, dt} = {\int \limits_{0}^{+ \infty} {\dfrac{\sin{(\lambda + 2) t}}{t} \, dt}} + {\int \limits_{0}^{+ \infty} {\dfrac{\sin{(\lambda - 2) t}}{t} \, dt}} $$
	
	Следует рассматривать несколько случаев:

	\begin{enumerate}
		\item $ \lambda > 2 $:	\\	
	Делая замену переменной, получим интегралы  типа (\ref{Gaussian_Integral}).
		\item $ \lambda = 2 $:	\\	
	Второй интеграл в правой части обнулится, а первый, путем заменый переменной, сведётся к (\ref{Gaussian_Integral}).
		\item $ \lambda  \in (-2;2) $:	\\	
	Очевидно, что $ I_4 = 0$.
		\item $ \lambda  = -2 $:	\\	
	Первый интеграл в правой части обнулится, а второй, путем заменый переменной, сведётся к (\ref{Gaussian_Integral}).
		\item $ \lambda < -2 $:	\\	
	Пользуясь тем, что функция $ \sin{x} $ --- нечетная, значит знак "минус" можно вынести и производя замену переменной, получим интегралы  типа (\ref{Gaussian_Integral}).
			
	\end{enumerate}
	
\end{enumerate}

Итого, в результате получается, что

\begin{equation} 
	F(\lambda) =    
    \left\{
    \begin{aligned}
    	\begin{array}{llc}
        0, & \lambda \in \{- \infty; 2\} \cup \{0\} \cup \{2; + \infty\} \\
        -\dfrac{\pi i}{2}, & \lambda \in (0; 2) \\
        \dfrac{\pi i}{2}, & \lambda \in (-2; 0) \\
        \dfrac{\pi i}{4}, & \lambda = -2 \\
        -\dfrac{\pi i}{4}, & \lambda = 2\\
		\end{array}   
    \end{aligned}
    \right. 
\end{equation}


\subsection{Вычисление~аналитического~преобразования~Фурье\\функции~\(f_2(t) = te^{-2t^2} \)} Преобразование Фурье $ F_2 (\lambda) $ функции $ f_2(t) = te^{-2t^2} $ задаётся формулой:
$$ F_2 (\lambda) = \int \limits_{- \infty}^{+ \infty} {te^{-2t^2} e^{-i\lambda t} \, dt}. $$

\noindent Выведем некоторую цепочку преобразований:

$$ \int \limits_{- \infty}^{+ \infty} {te^{-2t^2} e^{-i\lambda t} \, dt} = \int \limits_{- \infty}^{+ \infty} {te^{-2t^2 - i\lambda t} \, dt} = \intinf {te^{- \left (2t^2 + i \lambda t + \dfrac{i^2 \lambda ^2}{8} \right ) + \dfrac{i^2 \lambda ^2}{8}}} = $$

$$ = e^{- \dfrac{\lambda ^2}{8}} \intinf{te^{- \left (\sqrt{2}t + \dfrac{i \lambda}{2  \sqrt{2} }\right )^2}}{t} = \left \lbrace \sqrt{2}t + \dfrac{i \lambda}{2  \sqrt{2}} = s \right \rbrace = $$

$$ = \dfrac{1}{\sqrt{2}} e^{- \dfrac{\lambda ^2}{8}} \intinf{\left (\dfrac{s}{\sqrt{2}} - {\dfrac{i \lambda}{4}} \right )e^{- s^2}}{s} = \dfrac{1}{4} e^{- \dfrac{\lambda ^2}{8}} \intinf{s e^{- s^2}}{s} - \dfrac{i \lambda}{4 \sqrt{2}} e^{- \dfrac{\lambda ^2}{8}} \intinf{e^{- s^2}}{s} = $$

$$ = \dfrac{1}{4} e^{- \dfrac{\lambda ^2}{8}} \intinf{e^{- s^2}}{s^2} - \dfrac{i \lambda}{4 \sqrt{2}} e^{- \dfrac{\lambda ^2}{8}} \intinf{e^{- s^2}}{s}. $$

Таким образом, первый интеграл обнуляется, а второй является известным интегралом (\ref{Gaussian_Integral}). Следовательно получаем:

$$ F_2 (\lambda) = - \dfrac{i \sqrt{\pi}}{4 \sqrt{2}} \cdot \lambda \cdot e^{- \dfrac{\lambda ^2}{8}}. $$ \\

\subsection{Примеры, иллюстрирующие работу программы}

\begin{enumerate}
	\item Рассматрим функцию: 
	$$ f_1(t) = \dfrac{1 - \cos^2{t}}{t}. $$
	Заметим, что она имее разрыв первого рода в точке $ t = 0 $. Исходя из её аналитического представления, мы знаем, что у него пять точек разрыва ($ \lambda = \pm \dfrac{\pi i}{2}, \pm \dfrac{\pi i}{4}, 0 $).  
	
	Выберем несколько значений параметров $ \bigtriangleup t $ (шаг дискретизации), $ [a, b] $ (окно, ограничивающее область действия функции $ f_1(t) $), $ [c, d] $(окно для вывода преобразования Фурье $ F_1(\lambda) $). 
	
	\begin{enumerate}
		\item $ \bigtriangleup t = 0.9 $, $ [a, b] = [-30, 1000] $, $ [c, d] = [-2, 2] $.
		
		\includegraphics[width=1\linewidth]{21}
		
		\begin{corollary}
			\begin{itemize}
				\item Мнимая часть преобразования Фурье, полученные аналитически и численно, совпадают с точностью до ряби. Эта рябь неизбежно возникает в точках разрыва, но также она появляется и в точках непрерывности  функции $ F_1(\lambda) $. Первое связано со свойствами преобразования Фурье, а второе - с диапозоном окна.
				\item Из аналитического представления $ F_1(\lambda) $ следует, что вещественная часть преобразования Фурье равна нулю, в то время, как вещественная часть, полученная численно, по модудлю отлична от нуля не больше чем на шаг дискретизации $ \bigtriangleup t $. Отметим, что точки разрыва в преобразовании Фурье, посчитанного численно, веществнной и мнимой частях совпадают.
			\end{itemize}
		\end{corollary}
		
		\item $ \bigtriangleup t = 0.9 $, $ [a, b] = [-300, 2500] $, $ [c, d] = [-2, 2] $.
		
		\includegraphics[width=1\linewidth]{22}
		
		\begin{corollary}
			\begin{itemize}
				\item Мнимые части преобразования Фурье, полученные численно и аналитически совпадают. Но в отличии от предыдущего примера, рябь возникает только в точках разрыва (совсем ее убрать нельзя!). Следоватьельно увеличивая диапозон окна, можно добиться исчезновения ряби в точках неперерывности функции $ F_1(\lambda) $.
				\item Ситуация с действительной частью обстоит также, как и в предыдущем примере.
			
			\end{itemize}
		\end{corollary}
	\end{enumerate}

	\item Теперь рассмотрим функцию:	
		$$ f_2(t) = te^{-2t^2}. $$
		
		Как можно видеть, эта функция и её преобразование Фурье непрерывны.
		Выберем несколько значений параметров $ \bigtriangleup t $ (шаг дискретизации), $ [a, b] $ (окно, ограничивающее область действия функции $ f_1(t) $), $ [c, d] $(окно для вывода преобразования Фурье $ F_2(\lambda) $).  
		
		\begin{enumerate}
			\item $ \bigtriangleup t = 0.2 $, $ [a, b] = [-200, 500] $, $ [c, d] = [-2, 2] $.
			
			\includegraphics[width=1\linewidth]{12}
			
		\begin{corollary}
			\begin{itemize}
				\item Мнимые части преобразования Фурье, полученные численно и аналитически, совападают.
				\item Вещественная часть преобразования Фурье, полученная аналитически равна нулю, в то время, как, полученная численно, по модулю не превосходит шага дискретизации $ \bigtriangleup t $.
				\item Отсутсвует рябь. Т.к. нет точек разрыва.\\
			\end{itemize}
		\end{corollary}
		
			\item $ \bigtriangleup t = 1 $, $ [a, b] = [-60, 100] $, $ [c, d] = [-2, 2] $.
			
			\includegraphics[width=1\linewidth]{13}
			
			\begin{corollary}
				\begin{itemize}
					\item Вещественная часть преобразования Фурье по прежнему равна нулю, поэтому сдвиг ничего не изменит.
					\item Что касается мнимой части, то здесь наблюдается эффект наложения спектра. При этом численное преобразование Фурье не совпадает с аналитическим. Данное явление обусловлено нарушением соотношения $$ \bigtriangleup t \leq \dfrac{\pi}{\Lambda}, $$ где $ \bigtriangleup t $ --- шаг дискретизации, а $ \Lambda > 0 \, (|\lambda| < \Lambda)$ --- промежуток, где нужно устранить наложение спектра.\\
				В данном случае эффект наложения спектра можно устранить, т.к. функция $ F_1(\lambda) \rightarrow 0, $ при $ |\lambda| \rightarrow + \infty $. И чтобы его устранить нужно: увеличить окно $ \Rightarrow $ уменьшится шаг дискретизации $ \bigtriangleup t $ (см. пример выше);
			
				\end{itemize}
				
			\end{corollary}
			
		\end{enumerate}
		
		\item Далее рассмотрим функцию:	
		$$ f_3(t) = \dfrac{2}{1 + 3t^6} $$
		
		Данная функция является непрерывной. Будем вычилсять преобразование Фурье численно. Выберем несколько значений параметров $ \bigtriangleup t $ (шаг дискретизации), $ [a, b] $ (окно, ограничивающее область действия функции $ f_1(t) $), $ [c, d] $(окно для вывода преобразования Фурье $ F_3(\lambda) $).
		 
		\begin{enumerate}
			\item $ \bigtriangleup t = 0.1 $, $ [a, b] = [-40, 50] $, $ [c, d] = [-10, 10] $.
		
			\includegraphics[width=1\linewidth]{31}

			\item $ \bigtriangleup t = 0.1 $, $ [a, b] = [-200, 200] $, $ [c, d] = [-20, 20] $.		
		
			\includegraphics[width=1\linewidth]{32}
	
		\end{enumerate}
		
		\begin{corollary}
		Сравнинв графики мнимых и вещественных частей преобразования Фурье соответсвенно, мы делаем вывод, что ничего существенного не изменилось при подборе различных параметров. Но мы поняли по графику, что проеобразование Фурье непрерывное.
		\end{corollary}
		
		\item Наконец рассмотрим функцию:	
		$$ f_4(t) = e^{-5|t|}\ln({3 + t^4}) $$
		
		Эта функция - нерпрерывна. Будем вычилсять преобразование Фурье численно. Выберем несколько значений параметров $ \bigtriangleup t $ (шаг дискретизации), $ [a, b] $ (окно, ограничивающее область действия функции $ f_1(t) $), $ [c, d] $(окно для вывода преобразования Фурье $ F_4(\lambda) $).
		 
		\begin{enumerate}
			\item $ \bigtriangleup t = 0.1 $, $ [a, b] = [-100, 100] $, $ [c, d] = [-20, 20] $.
		
			\includegraphics[width=1\linewidth]{41}

			\item $ \bigtriangleup t = 0.1 $, $ [a, b] = [-20, 50] $, $ [c, d] = [-10, 10] $.		
		
			\includegraphics[width=1\linewidth]{42}
	
		\end{enumerate}
		
		\begin{corollary}
		Сравнинв графики мнимых и вещественных частей преобразования Фурье соответсвенно, мы делаем вывод, что ничего существенного не изменилось при подборе различных параметров. И опять мы понимаем, что преобразование Фурье непрерывно.
		\end{corollary}

\end{enumerate}

\end{document}